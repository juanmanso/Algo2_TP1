\input{.command.tex}
% En el siguiente archivo se configuran las variables del trabajo práctico
%% \providecommand es similar a \newcommnad, salvo que el primero ante un 
%% conflicto en la compilación, es ignorado.

% Al comienzo de un TP se debe modificar los argumentos de los comandos


\providecommand{\myTitle}{TRABAJO PRÁCTICO Nº1}
\providecommand{\mySubtitle}{Transformada de Fourier}

\providecommand{\mySubject}{Algoritmos y Programación II (95.12)}
\providecommand{\myKeywords}{UBA, Ingeniería, C++, 95.12, Algoritmos y Programación}

% No es necesario modificar este
\providecommand{\myHeaderLogo}{header_fiuba}

\providecommand{\myAuthorSurname}{Andreasen & Manso}
\providecommand{\myTimePeriod}{Año 2016 - 2\textsuperscript{do} Cuatrimestre}


% Crear los integrantes del TP con el comando \PutMember donde
%%		1) Apellido, Nombre
%%		2) Número de Padrón
%%		3) E-Mail (Si el mail contiene '_', escribirlos como '\_'
\providecommand{\CoverMembers}[0]
{		\PutMember{Andreasen, Ricardo} {96322} {ra\_95\_1@hotmail.com} 
		\PutMember{Manso, Juan} {96133} {juanmanso@gmail.com}
}

\providecommand{\myLstLanguage}{C++}

\Pagebreakfalse		% Setea si hay un salto de página en la carátula
\Indexfalse
\Siunitxfalse		% Si quiero utilizar el paquete, \siunixtrue. Si no \siunitxfalse
\Listingstrue		% Idem con paquete listings (programación)
\Keywordsfalse
